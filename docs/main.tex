\documentclass{achemso}

\usepackage{physics}
\usepackage{amsmath}
\usepackage{verbatim}
\usepackage{listings}
\usepackage{xcolor}
\definecolor{bg}{RGB}{240,240,240}
\lstset{
  basicstyle=\linespread{1.0}\ttfamily,
  backgroundcolor=\color{bg},
  }

\author{Brent R. Westbrook}
\title{qcrab documentation}
\date{}

% must be used in math mode
\newcommand\eab{E_{\vb A \vb B}}

\begin{document}

\maketitle

\section{Overlap Integrals}
\label{sec:overlap}

As shown in Ref. \citenum{ho12}, the expression for an overlap integral of two
primitive Gaussian functions, $S$, is given by:

\begin{equation}
  \label{eq:s}
  S = \eab S_x S_y S_z,
\end{equation}

\noindent
where $\eab = \exp[-\frac{\alpha\beta}{\alpha + \beta}|\vb A - \vb B|^2]$ and
$S_x$, for example, is given by:

\begin{equation}
  \label{eq:sx}
  S_x = \sqrt{\frac{\pi}{\alpha + \beta}}
  \sum_{i_x = 0}^{a_x}\sum_{j_x = 0}^{b_x}
  \binom{a_x}{i_x} \binom{b_x}{j_x}
  \frac{(i_x + j_x + 1)!!}{[2(\alpha + \beta)]^{(i_x + j_x)/2}}
  (P_x - A_x)^{a_x - i_x}
  (P_x - B_x)^{b_x - j_x}
\end{equation}

\noindent
As a note, $|\vb A - \vb B|^2$ is the magnitude of $\vb A - \vb B$
($\sqrt{\sum_i (a_i - b_i)^2}$) squared ($\sum_i (a_i - b_i)^2$), or
$(\vb A - \vb B) \cdot (\vb A - \vb B)$.

In all of these expressions, $\alpha$ is the exponent for the first primitive
Gaussian function, $\beta$ is the exponent for the second primitive Gaussian,
$\vb P = \frac{\alpha \vb A + \beta \vb B}{\alpha + \beta}$, $\vb A$ and $\vb B$
are the origins of the basis functions (the Cartesian coordinates of the nuclei
they are centered on), and $a_x$ and $b_x$ are the $x$ components of the angular
momentum for the Gaussian functions. For example, in an $s$ orbital, $l = 0$,
and $l_x$, $l_y$, and $l_z$ are given by the tuple (0, 0, 0), while for a $p$
orbital, there are three combinations for which $l = 1 = l_x + l_y + l_z$: (1,
0, 0), (0, 1, 0), and (0, 0, 1). Consequently, for shells composed of multiple
primitive Gaussian functions, we must loop over these combinations, as well as
the primitives.

In the code for \verb|overlap_ints|, I first loop over the shells and collect
these possibilities, as shown in Listing~\ref{lst:ls}.

\begin{lstlisting}[caption={Collecting $l$ values for shells}, label={lst:ls}]
let mut ls = Vec::new();
let mut ss = Vec::new();
for (i, shell) in self.0.iter().enumerate() {
    let l = match shell.contr[0].l {
        0 => vec![(0, 0, 0)],
        1 => vec![(1, 0, 0), (0, 1, 0), (0, 0, 1)],
        2 => vec![
            (2, 0, 0),
            (0, 2, 0),
            (0, 0, 2),
            (0, 1, 1),
            (1, 1, 0),
            (1, 0, 1),
        ],
        _ => panic!("unmatched l value {}", shell.contr[0].l),
    };
    let s = vec![i; l.len()];
    ls.extend(l);
    ss.extend(s);
}
\end{lstlisting}

With the combinations of angular momenta stored in \verb|ls| and the shells
associated with them in \verb|ss|, we can iterate over them, effectively
iterating over the orbitals themselves. For each combination of orbitals, we
compute the overlap integral for the orbitals, $S_{ij}$, as the sum over
primitives, where $N$ is the number of primitives in shell $i$, and $M$ is the
number of primitives in shell $j$:

\begin{equation}
  \label{eq:sij}
  S_{ji} = S_{ij} = \sum_n^N\sum_m^M c_n c_m E_{\vb A \vb B} S_x S_y S_z
\end{equation}

Recall that $\eab$ includes $\alpha$ and $\beta$, which are subscripted by $n$
and $m$ in this sum, so $\eab$ does depend on $n$ and $m$. Similarly, $\alpha$
and $\beta$ in Eqn.~\ref{eq:sx} are $\alpha_n$ and $\beta_m$, and $\vb P$ must
also be computed on each iteration, again because of its dependence on $\alpha$
and $\beta$.

\bibliography{refs}

\end{document}